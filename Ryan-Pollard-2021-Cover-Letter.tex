%!TEX TS-program = xelatex
%!TEX encoding = UTF-8 Unicode
% Awesome CV LaTeX Template for CV/Resume
%
% This template has been downloaded from:
% https://github.com/posquit0/Awesome-CV
%
% Author:
% Claud D. Park <posquit0.bj@gmail.com>
% http://www.posquit0.com
%
%
% Adapted to be an Rmarkdown template by Mitchell O'Hara-Wild
% 23 November 2018
%
% Template license:
% CC BY-SA 4.0 (https://creativecommons.org/licenses/by-sa/4.0/)
%
%-------------------------------------------------------------------------------
% CONFIGURATIONS
%-------------------------------------------------------------------------------
% A4 paper size by default, use 'letterpaper' for US letter
\documentclass[11pt, a4paper]{awesome-cv}

% Configure page margins with geometry
\geometry{left=1.4cm, top=.8cm, right=1.4cm, bottom=1.8cm, footskip=.5cm}

% Specify the location of the included fonts
\fontdir[fonts/]

% Color for highlights
% Awesome Colors: awesome-emerald, awesome-skyblue, awesome-red, awesome-pink, awesome-orange
%                 awesome-nephritis, awesome-concrete, awesome-darknight

\definecolor{awesome}{HTML}{414141}

% Colors for text
% Uncomment if you would like to specify your own color
% \definecolor{darktext}{HTML}{414141}
% \definecolor{text}{HTML}{333333}
% \definecolor{graytext}{HTML}{5D5D5D}
% \definecolor{lighttext}{HTML}{999999}

% Set false if you don't want to highlight section with awesome color
\setbool{acvSectionColorHighlight}{true}

% If you would like to change the social information separator from a pipe (|) to something else
\renewcommand{\acvHeaderSocialSep}{\quad\textbar\quad}

\def\endfirstpage{\newpage}

%-------------------------------------------------------------------------------
%	PERSONAL INFORMATION
%	Comment any of the lines below if they are not required
%-------------------------------------------------------------------------------
% Available options: circle|rectangle,edge/noedge,left/right

\photo{rypoll.jpg}
\name{Ryan Pollard \textbar{} Cover Letter}{}

\position{Data Scientist}
\address{Manchester, United Kingdom}

\mobile{+44 7500 693 434}
\email{\href{mailto:rypoll@gmail.com}{\nolinkurl{rypoll@gmail.com}}}
\github{rypoll}
\linkedin{rypoll}

% \gitlab{gitlab-id}
% \stackoverflow{SO-id}{SO-name}
% \skype{skype-id}
% \reddit{reddit-id}


\usepackage{booktabs}

\providecommand{\tightlist}{%
	\setlength{\itemsep}{0pt}\setlength{\parskip}{0pt}}

%------------------------------------------------------------------------------



% Pandoc CSL macros
\newlength{\cslhangindent}
\setlength{\cslhangindent}{1.5em}
\newlength{\csllabelwidth}
\setlength{\csllabelwidth}{3em}
\newenvironment{CSLReferences}[3] % #1 hanging-ident, #2 entry spacing
 {% don't indent paragraphs
  \setlength{\parindent}{0pt}
  % turn on hanging indent if param 1 is 1
  \ifodd #1 \everypar{\setlength{\hangindent}{\cslhangindent}}\ignorespaces\fi
  % set entry spacing
  \ifnum #2 > 0
  \setlength{\parskip}{#2\baselineskip}
  \fi
 }%
 {}
\usepackage{calc}
\newcommand{\CSLBlock}[1]{#1\hfill\break}
\newcommand{\CSLLeftMargin}[1]{\parbox[t]{\csllabelwidth}{#1}}
\newcommand{\CSLRightInline}[1]{\parbox[t]{\linewidth - \csllabelwidth}{#1}}
\newcommand{\CSLIndent}[1]{\hspace{\cslhangindent}#1}

\begin{document}

% Print the header with above personal informations
% Give optional argument to change alignment(C: center, L: left, R: right)
\makecvheader

% Print the footer with 3 arguments(<left>, <center>, <right>)
% Leave any of these blank if they are not needed
% 2019-02-14 Chris Umphlett - add flexibility to the document name in footer, rather than have it be static Curriculum Vitae
\makecvfooter
  {December, 2021}
    {Ryan Pollard \textbar{} Cover Letter~~~·~~~Curriculum Vitae}
  {\thepage}


%-------------------------------------------------------------------------------
%	CV/RESUME CONTENT
%	Each section is imported separately, open each file in turn to modify content
%------------------------------------------------------------------------------



\hypertarget{about-me}{%
\section{About me}\label{about-me}}

I'm a highly motivated person with a strong desire to use the skills
I've gained and the skills I'm continuing to learn in order to provide
benefit to organisations and individuals. With over seven years of
experience across different industries, I've been able to learn many
different aspects of how data is used in a practical setting - from
dashboards, data analysis, machine learning models and deployment of
models into apps. I'm committed to improving my own skills by learning
new data science techniques on a daily basis; I recently taught myself
NLP techniques such as sentiment analysis and recommendation systems and
went on to build my own personal projects and be involved in projects
with clients that used these acquired skills.

Above all, I'm someone who places extreme importance on doing a great
job. With my recent work on the freelance website Upwork I've managed to
be involved in interesting and challenging projects and managed to
receive 5-star reviews for every single project I've taken on. I like
being in situations where I'm faced with new problems and finding ways
to deal with them in order to build something functional and useful. I
spent a lot of time learning my craft and continue to do so and the
motivator behind that is having the ability to aid and provide function
to people and organisations.

\hypertarget{background}{%
\section{Background}\label{background}}

I'm a qualified Statistician with over six years experience in various
data science roles.

My learning journey started at the University of Manchester where I
studied Mathematics and naturally gravitated towards statistical
modules. After university, I worked as a Data Scientist in the
Pharmaceutical, Energy and Banking industries; along the way using many
different statistical techniques and seeing how machine learning can
provide huge benefits to an organisation.

I started my career in a Pharmaceutical company - performing statistical
tests and analysing medical questionnaires in order to provide insight
and draw conclusions. I processed and performed tests that would later
go onto publications. These publications would be used to highlight the
relationship between drugs and disease outcomes.

Later, I moved into the energy industry, building predictive models that
were used to segment customers in order for the business to react to
it's client base in the most effective way possible. I built statistical
models using random forests to classify customers. The classifications
would drive business decisions, for instance, a model would identify
certain customers who are in debt and would be best targeted by a phone
call as a preferred method of communication in order to recoup that
debt. Understanding customer behavior using the models I created made
processes more efficient and saved money in turn.

Moving onto the banking industry, I was in charge of the maintenance and
updating of credit risk models that dictated the allocation of funds of
the bank. I also created dashboards -- producing clear and concise
analysis that ensured the business understood its current portfolio and
make decisions off the back of that.

More recently I took time out to study a Masters in Statistics to
understand my profession better. I've taken on freelance projects -
building machine learning models and deploying them into apps for
various clients.

After a stint of teaching Mathematics and traveling in 2018, I chose to
study a MSc in Statistics in 2020 at the University of Sheffield to
strengthen the foundation of the Data Science techniques I've learnt and
also acquire modern techniques. I used various classification and
dimension reduction here, using sci-kit learn and keras, Bayesian
statistics and Time series. I was also involved in multiple projects
making R shiny dashboards.

All these projects can be found on my github, please feel free to take a
look using the link at the top of this page.

\hypertarget{what-i-can-do}{%
\section{What I can do}\label{what-i-can-do}}

Throughout my career I've used and understood many different techniques
and programming languages. I've used Python, SQL, SAS and R extensively.
On top of that I have a depth of experience in various Machine Learning
techniques and other statistical methods that can provide benefits in
different ways.

In my degree and freelancing career I've built various machine learning
models (sci-kit learn, Neural Networks and more), to classify, reduce
dimensionality and deploy into apps through which users without a
statistical background could investigate the profile of genes.

I'm passionate about using my skills to provide functional solutions to
people and organisations.



\end{document}
